%%%%%%%%%%%%%%%%%%%%%%%%%%%%%%%%%%%%%%%%%%%%%%%%%%%%%%%%%%%%%%%%%%
%%%%%%%% ICML 2013 EXAMPLE LATEX SUBMISSION FILE %%%%%%%%%%%%%%%%%
%%%%%%%%%%%%%%%%%%%%%%%%%%%%%%%%%%%%%%%%%%%%%%%%%%%%%%%%%%%%%%%%%%

% Use the following line _only_ if you're still using LaTeX 2.09.
%\documentstyle[icml2013,epsf,natbib]{article}
% If you rely on Latex2e packages, like most moden people use this:
\documentclass{article}

% For figures
\usepackage{graphicx} % more modern
%\usepackage{epsfig} % less modern
% \usepackage{subfigure}
\usepackage{subcaption}
\usepackage{multicol}

% For citations
\usepackage{natbib}

% For algorithms
\usepackage{algorithm}
\usepackage{algorithmic}

% For math
\usepackage{amsmath}
\usepackage{siunitx}

% As of 2011, we use the hyperref package to produce hyperlinks in the
% resulting PDF.  If this breaks your system, please commend out the
% following usepackage line and replace \usepackage{icml2013} with
% \usepackage[nohyperref]{icml2013} above.
\usepackage{hyperref}

% Packages hyperref and algorithmic misbehave sometimes.  We can fix
% this with the following command.
\newcommand{\theHalgorithm}{\arabic{algorithm}}

% Employ the following version of the ``usepackage'' statement for
% submitting the draft version of the paper for review.  This will set
% the note in the first column to ``Under review.  Do not distribute.''
\usepackage{icml2013}
% Employ this version of the ``usepackage'' statement after the paper has
% been accepted, when creating the final version.  This will set the
% note in the first column to ``Proceedings of the...''
% \usepackage[accepted]{icml2013}


% The \icmltitle you define below is probably too long as a header.
% Therefore, a short form for the running title is supplied here:
\icmltitlerunning{6.867: Homework 3}

\begin{document}

\twocolumn[
  \icmltitle{6.867: Homework 3}

  % % It is OKAY to include author information, even for blind
  % % submissions: the style file will automatically remove it for you
  % % unless you've provided the [accepted] option to the icml2013
  % % package.
  % \icmlauthor{Your Name}{email@yourdomain.edu}
  % \icmladdress{Your Fantastic Institute,
  %             314159 Pi St., Palo Alto, CA 94306 USA}
  % \icmlauthor{Your CoAuthor's Name}{email@coauthordomain.edu}
  % \icmladdress{Their Fantastic Institute,
  %             27182 Exp St., Toronto, ON M6H 2T1 CANADA}

  % You may provide any keywords that you
  % find helpful for describing your paper; these are used to populate
  % the "keywords" metadata in the PDF but will not be shown in the document
  \icmlkeywords{boring formatting information, machine learning, ICML}

  \vskip 0.3in
]

\section{Neural Networks}
In this section, we explore logistic regression with L1 and L2 regularization. We use gradient descent to compare the resulting weight vectors under different regularizers and regularization parameters, and we evaluate the effect of these choices in the context of multiple data sets.

\subsection{ReLU + Softmax}


\subsection{Initialization}
This is a reasonable choice because $m$ represents the number of input neurons in each of our layers, which is what should control the variance of the distribution. The reason we do this is to ensure that the random input to a unit in a layer is agnostic to the number of inputs it receives.

\subsection{Regularization}
This would impact the pseudocode because now we must also incorporate this additional regularization term when doing the stochastic gradient descent update. Specifically, the update step before regularization was:

$$\theta \leftarrow \theta - \eta \frac{\delta l(y,F(x;\theta))}{\delta \theta}$$

With regularization, the update step now becomes:

$\theta \leftarrow \theta - \eta \frac{\delta J}{\delta \theta}$ where $J(\theta) = L(\theta) + R(\theta)$ and $R(\theta) = \lambda *(\sum_{ij} w_{ij}^{(1)}^2 + \sum_{ij} w_{ij}^{(2)}^2)$. Furthermore, $J'(\theta) = L'(\theta) + R'(\theta)$. $L'(\theta)$ has already been calculated and $R'(\theta) = 2* \lambda * \sum_{ij} w_{ij}$

\subsection{Binary classification}


\subsection{Multi-class classification}



\section{Convolutional Neural Networks}
In this section, we explore various versions of the dual form of support vector machines, first with slack variables and then with generalized kernel functions.

\subsection{Convolutional filter receptive field}

\subsection{Run the Tensorflow conv net}

\subsection{Add pooling layers}

\subsection{Regularize your network!}

\subsection{Experiment with your architecture}

\subsection{Optimize your architecture}

\subsection{Test your final architecture on variations of the data}




\end{document}


% This document was modified from the file originally made available by
% Pat Langley and Andrea Danyluk for ICML-2K. This version was
% created by Lise Getoor and Tobias Scheffer, it was slightly modified
% from the 2010 version by Thorsten Joachims & Johannes Fuernkranz,
% slightly modified from the 2009 version by Kiri Wagstaff and
% Sam Roweis's 2008 version, which is slightly modified from
% Prasad Tadepalli's 2007 version which is a lightly
% changed version of the previous year's version by Andrew Moore,
% which was in turn edited from those of Kristian Kersting and
% Codrina Lauth. Alex Smola contributed to the algorithmic style files.
